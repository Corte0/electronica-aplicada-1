\chapter{Conclusiones}

\section{Analisis de mediciones}

\subsection{Mediciones con y sin regulador LM317}

Pudimos observar durante las primeras mediciones realizadas, en las cuales se encontraba la fuente sin regulador, se obtuvo un factor de
ripple de $3{,}0258 \percent$ y una caida de tension aproximada de 10V en el punto de mayor demanda, lo cual evidencia una tension de salida poco
constante y dependiente de la carga.

Por otra parte, las mediciones realizadas con el regulador incorporado, se pudo apreciar una mejora significativa en la calidad de la
tension de salida. Se midio un ripple de $353{,}5 \uV$ , con un factor de ripple inferior al $0,003 \percent$ valores considerablemente
mejores que los estimados inicialmente y plenamente satisfactorios para los requerimientos del proyecto.

\subsection{Complicaciones termicas}

Sin embargo, en el aspecto termico no obtuvimos resultados optimos. Bajo condiciones de maxima carga, el LM317 alcanzo una
temperatura de juntura estimada de $135\Celsius$, a pesar de la incorporacion de grasa disipadora. Este valor se aproxima al limite termico del
componente, por lo que en futuras implementaciones lo ideal seria trabajar con el regulador y un disipador termico de mayor tamano o
incorporar una fuente de disipacion forzada.
